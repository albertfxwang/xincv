%: vim: formatoptions+=tw inde=
% The evolving curriculum vitae of Xin Wang <albertfxwang@gmail.com>
% based upon the cv of Aleksander Morgado <aleksander@gnu.org>
% itself based on Alessandro Plasmati's cv template:
% http://www.scribd.com/doc/16335667/Writing-your-Professional-CV-with-LaTeX
%--------------------------------------------------------------------------------------------------------------------------------
\documentclass[letterpaper,10pt]{article}

\usepackage{geometry}
\newcommand{\marg}{0.9in}
\geometry{letterpaper,tmargin=\marg,bmargin=\marg,lmargin=\marg,rmargin=\marg,headheight=0in,headsep=.3in,footskip=.3in}
%\geometry{letterpaper,tmargin=1cm,bmargin=1cm,lmargin=0.5cm,rmargin=1cm,headheight=0in,headsep=0in,footskip=0in}
% US letter layout
%\setlength{\paperwidth}{8.5in}
%\setlength{\paperheight}{11in}
% A4 paper layout
%\setlength{\paperwidth}{21cm}
%\setlength{\paperheight}{29.7cm}

%\usepackage[latin1]{inputenc}
\usepackage{marvosym}
\usepackage{fontspec} 					    %for loading fonts
\usepackage{xunicode,xltxtra,url,parskip} 	%other packages for formatting
\RequirePackage{color,graphicx}
\usepackage[usenames,dvipsnames]{xcolor}
%\usepackage{fullpage}				        % to fix t/b/l/r margins
\usepackage{textcomp}
\usepackage{titlesec}					    %custom \section
\usepackage{footmisc}
\usepackage{longtable}                      %for long tables which can be break between pages

%Setup hyperref package, and colours for links
\usepackage{hyperref}
\definecolor{linkcolour}{rgb}{0,0.2,0.6}
\hypersetup{colorlinks,breaklinks,urlcolor=linkcolour, linkcolor=linkcolour}

%FONTS
\defaultfontfeatures{Mapping=tex-text}

\renewcommand*{\footnotelayout}{\normalsize}

\titleformat{\section}{\large\scshape\raggedright}{}{0em}{}[\titlerule]

\titlespacing{\section}{0pt}{2pt}{1pt}

%\addtolength{\parskip}{\baselineskip}		space prob occurs when this is set free
\setlength{\parindent}{0 em}
%\setlength{\parskip}{-1em}

%= = = = = = = = = = = = = = = = = = = = = = = = = = = = = = = = = = = = = = = =
% add page number without the header line
\usepackage{fancyhdr}
% Redefine plain style, which is used for titlepage and chapter beginnings
% From https://tex.stackexchange.com/a/30230/828
\fancypagestyle{plain}{%
    \renewcommand{\headrulewidth}{0pt}%
    \fancyhf{}% Start with clearing everything in the header and footer
    \fancyfoot[C]{\thepage}% Set the center of the footer to be the page number, other options: L, R
}
\pagestyle{plain}


%%%%%%%%%%%%%%%%%%%%%%%%%%%%%%%%%%%%%%%%%%%%%%%%%%%%%%%%%%%%%%%%%%%%%%%%%%%%%%
\begin{document}

\par{\centering
  {\huge \textsc{Xin Wang}}
%  \\ Telecommunications Engineer, MSc. \bigskip    sub-title
\par}

%%%%%%%%%%%%%%%%%%%%%%%%%%%%%%%%%%%%%%%%%%%%%%%%%%%%%%%%%%%%%%%%%%%%%%%%%%%%%
\section{Personal Information}
\begin{tabular}{rl}
%    \textsc{Date and Place of Birth:}       & September 1$^{\rm st}$, 1988  |  Tianjin, China \\
    \textsc{Current Status:}  & Graduate Student at University of California, Los Angeles \\
    \textsc{E-mail, Cell and Skype Account:}  & \href{mailto:albertfxwang@gmail.com}{albertfxwang@gmail.com}  |  +1-805-574-0025  |  albertfxwang \\
    \textsc{Mailing Address:} & 3251 S. Sepulveda Blvd., Apt. 307, Los Angeles, CA 90034, USA \\
\end{tabular}

%%%%%%%%%%%%%%%%%%%%%%%%%%%%%%%%%%%%%%%%%%%%%%%%%%%%%%%%%%%%%%%%%%%%%%%%%%%%%%
%\vspace{-0.5cm}
%\section{Objective}
%\textbf{Admission to Ph.D. Program in Astrophysics (Fall 2013) at University of California, Santa Barbara}

%%%%%%%%%%%%%%%%%%%%%%%%%%%%%%%%%%%%%%%%%%%%%%%%%%%%%%%%%%%%%%%%%%%%%%%%%%%%%%
\section{Education}
\begin{tabular}{r|p{6.5in}}
%\begin{tabular}{r|p{17.5cm}}
  %--------------------------------------------------------
  \textsc{Sept. 2015}--      &   Department of Physics and Astronomy, UCLA    |   \textbf{Towards Ph.D. in Astrophysics (Jun. 2019)}   \\
  \textsc{Present}
  & Field of Interest: Spatially Resolved Spectroscopy, Chemical Evolution of 
  Galaxies, \\ 
  & ~~~~~~~~~~~~~~~~~~~~~~ Extragalactic Nebular Emission, Strong Gravitational Lensing. \\
  &   \texttt{Advisor: Prof. Tommaso Treu}     \\
  \multicolumn{2}{c}{} \\
  %--------------------------------------------------------
  \textsc{Sept. 2013}--      &   Physics Department, University of California, Santa Barbara    |   \textbf{M.A. in Physics (Jun. 2015)}   \\
  \textsc{Sept. 2015}
  &   \texttt{Advisor: Prof. Tommaso Treu}; \quad Cumulative Total (Grad) GPA: 3.96     \\
  \multicolumn{2}{c}{} \\
  %--------------------------------------------------------
  \textsc{Sept. 2010}--     &   School of Astronomy and Space Sciences, Nanjing University  |  \textbf{M.Sc. in Astrophysics (Jun. 2013)}   \\
  \textsc{May 2013}
  & Field of Interest: Precision Cosmology, Galaxy Clusters, Gamma-ray Bursts. \\
  & \texttt{Advisors: Profs. Y. F. Huang, Charling Tao, Gong-Bo Zhao}   \\
  \multicolumn{2}{c}{} \\
  %--------------------------------------------------------
  \textsc{Sept. 2006}--     &   Department of Astronomy, Nanjing University  |  \textbf{B.Sc. in Astronomy (Jun. 2010)}    \\
  \textsc{Jun. 2010}       &   Weighted Average Score: 84.64/100 (overall), 87.68/100 (major); Ranking: 2$^{\rm nd}$/26  \\
  %========================================================
\end{tabular}

%%%%%%%%%%%%%%%%%%%%%%%%%%%%%%%%%%%%%%%%%%%%%%%%%%%%%%%%%%%%%%%%%%%%%%%%%%%%%%
\vspace{1em}
\section{Research Experience}
\vspace{-1ex}
\begin{longtable}{r|p{5.5in}}
%\begin{longtable}{r|p{17cm}}
  %--------------------------------------------------------
  \textsc{Sept. 2013}--  &   \emph{Title: The Grism Lens-Amplified Survey from Space (GLASS) program}   \\
  \textsc{Present}       &   \small{\textit{GLASS} is a cycle-21 HST Large Program allocated 140 orbits of Grism spectroscopy
  assisted with HST optical and infrared imaging. We survey the core and infall regions of 10 dynamically relaxed, massive
  clusters, including 8 targeted by CLASH and 6 Frontier Fields. We will address three scientific questions: 1) What's the role that
  galaxies play in the process of reionization? 2) Why and how is galaxy evolution environmental dependent? 3) How do metals cycle
  in and out of galaxies and what's the interplay between cycling of metals and SF activities?}   \\
  & \textbf{Project in progress and scientific product: 
        \hyperlink{17.wang.apj}{Wang et al. (2017)},
        \hyperlink{15.wang.apj}{Wang et al. (2015)},
        \hyperlink{15.jones.aj}{Jones et al. (2015)}} \\
  \multicolumn{2}{c}{} \\
  %--------------------------------------------------------
%  \textsc{Sept. 2012}--  &   \emph{Title: Applications of the Non-Parametric Bayesian Method to Cosmological Reconstructions}	\\
%  \textsc{Aug. 2013}     &   \small{We attempt to use the newly proposed non-parametric Bayesian method to reconstruct the
%  evolution history of some crucial cosmological quantities, i.e., primordial perturbation power spectrum, dark energy equation of
%  state, growth function. We believe that some unique abrupt features of the primordial power spectrum can be revealed by our
%  method. With the help of Fisher formalism and principle component analysis, it is straightforward to forecast the possibility
%  that those features can be detected by next generation projects, i.e., AS3, BigBOSS, Euclid, CMBPol.}	\\
%  &   \textbf{Scientific Product: paper to be submitted} \\
%  \multicolumn{2}{c}{} \\
  %--------------------------------------------------------
  \textsc{Jan. 2012}--   &   \emph{Title: Constraints on Cosmic Neutrinos and Dark Energy Revisited}  \\
  \textsc{Dec. 2012}     &   \small{Using various cosmological observations, i.e., CMB, weak lensing (WL), BAO, observational Hubble
  parameter data (OHD), type Ia supernovae (SNIa), we impose constraints on the sum of neutrino masses ($\Sigma m_{\nu}$), the
  effective number of neutrino species ($N_{\rm eff}$) and dark energy equation of state ($w$). We find that a tight upper limit
  on $\Sigma m_{\nu}$ can be extracted if $N_{\rm eff}$ and $w$ are fixed, however it will be severely weakened if $N_{\rm eff}$ and
  $w$ are allowed to vary. This result raises questions on the robustness of previous strict upper bounds on $\Sigma m_{\nu}$,
  reported in the literature.
  %The best-fits from our most generalized constraint read $\Sigma m_{\nu}=0.556^{+0.231}_{-0.288}\rm eV$, $N_{\rm eff}=3.839\pm0.452$, and $w=-1.058\pm0.088$. 
  The different constraining abilities of current WL, OHD and SNIa samples are assessed and compared.}\\
  &   \textbf{Scientific Product: \hyperlink{12.wang.jcap}{Wang et al. (2012)}}  \\
  \multicolumn{2}{c}{} \\
  %--------------------------------------------------------
%  \textsc{Sept. 2011}--  &   \emph{Title: Using the Cosmic Distance-Duality Relation to Test the $f_{\rm gas}$ Measurements in Galaxy Clusters} \\
%  \textsc{Jan. 2012}     &   \small{We propose a new method to assess X-ray measurements of galaxy cluster gas fraction ($f_{\rm
%  gas}$), via a combination of the Union2 SNIa compilation and the cosmic distance-duality relation, $\eta_{\rm{theory}}=D_{\rm
%  L}(1+z)^{-2}/D_{\rm A}=1$. Since in all previous estimations, $\eta_{\rm{theory}}=1$ is readily assumed, we use this constraint to
%  recover the cosmological information from a given set of $f_{\rm gas}$ data. Our results show that the $f_{\rm gas}$ sample of
%  Ettori et al. 2009 is endowed with an $\Omega_{\Lambda}=0$ reference cosmology, rather than the reported model
%  ($\Omega_\textrm{m}=0.3,\Omega_{\Lambda}=0.7$), which is excluded even at 3-$\sigma$ confidence level.}\\
%  &   \textbf{Scientific Product: \hyperlink{13.wang.raa}{Wang et al. (2013)}}  \\
%  \multicolumn{2}{c}{} \\
  %--------------------------------------------------------
%  \textsc{Jan. 2011}--   &   \emph{Title: Using the Test of the Distance-Duality Relation to Probe the Morphology of Galaxy Clusters} \\
%  \textsc{Oct. 2011}     &   \small{Aiming at probing the intrinsic morphology of galaxy clusters, we make a cosmological
%  model-independent test of the cosmic distance-duality relation, $D_{\rm L}(1+z)^{-2}/D_{\rm A}=1$, by two new methods. The
%  luminosity distances ($D_{\rm L}$) are obtained from the Union2 SNIa compilation. The angular diameter distances ($D_{\rm A}$) are
%  provided by two cluster morphological models, which are elliptical $\beta$-model and spherical $\beta$-model. Our results support
%  that the marked triaxial ellipsoidal model is a better geometrical hypothesis describing the structure of galaxy clusters compared
%  with the spherical $\beta$-model.}    \\
%  &   \textbf{Scientific Product: \hyperlink{12.meng.apj}{Meng et al. (2012)}} \\
%  \multicolumn{2}{c}{} \\
  %--------------------------------------------------------
  \textsc{Sept. 2008}--  &   \emph{Title: Investigation on the Emission from the Receding Jet of Gamma-Ray Bursts}   \\
  \textsc{Jun. 2010}    &   \small{We studied the dynamical evolution of double-sided jets launched by 
  the central engine of GRBs and calculated the afterglow emission from both jet components. For the first time, we present a 
  detailed numerical study on the afterglow contributed from the jet component receding from the observer, with the effects of 
  synchrotron self-absorption and equal arrival time surface taken into account. It is found that the receding jet emission is 
  generally very weak and only manifests as a plateau in the late time radio afterglow light curves. However the emission from the 
  receding jet can be significantly enhanced and possibly detectable, if the circum-burst medium density is high.} \\
  &   \textbf{Scientific Product: \hyperlink{09.wang.aa}{Wang et al. (2009)}, \hyperlink{10.wang.scichina}{Wang et al. (2010)}} \\
  %========================================================
\end{longtable}

%%%%%%%%%%%%%%%%%%%%%%%%%%%%%%%%%%%%%%%%%%%%%%%%%%%%%%%%%%%%%%%%%%%%%%%%%%%%%%
%\vspace{1em}
\section{Publications}
\begin{longtable}{p{6in}}
%\begin{longtable}{p{17cm}}
\multicolumn{1}{c}{\textsc{1st/2nd Author Papers in Refereed Academic Journals}}      \\
\vspace{-0.2cm}
\begin{list}{\labelitemi}{\leftmargin=0.5em}
    \item[1]\hypertarget{09.wang.aa}{} \textbf{Wang, X.}, Huang, Y. F., \& Kong, S. W. On the Afterglow from the Receding Jet of Gamma-Ray Bursts. 2009, \textit{Astron. Astrophys.}, 505, 1213 (\href{http://arxiv.org/abs/0903.3119}{arXiv:0903.3119})
    \item[2]\hypertarget{10.wang.scichina}{} \textbf{Wang, X.}, Huang, Y. F., \& Kong, S. W. Constraint on the Counter-jet Emission in GRB Afterglows from GRB 980703. 2010, \textit{Sci. China-Phys. Mech. Astron.}, 53 (Suppl.1), 259
    \item[3]\hypertarget{12.wang.jcap}{} \textbf{Wang, X.}, Meng, X.-L. et al. Observational Constraints on Cosmic Neutrinos and Dark Energy Revisited. 2012, \textit{J. Cosmol. Astropart. Phys.}, 11, 018 (\href{http://arxiv.org/abs/1210.2136}{arXiv:1210.2136})
    \item[4]\hypertarget{13.wang.raa}{} \textbf{Wang, X.}, Meng, X.-L., \& Huang, Y. F., Testing X-ray Measurements of Galaxy Cluster Gas
  Mass Fraction Using the Cosmic Distance-Duality Relation and Type Ia Supernovae. 2013, RAA, 13, 1013 (\href{http://arxiv.org/abs/1305.2077}{arXiv:1305.2077})
    \item[5]\hypertarget{15.jones.aj}{} Jones, T., \textbf{Wang, X.} et al. The Grism Lens-Amplified Survey from Space (GLASS) II.
        Gas-Phase Metallicity and Radial Gradients in an Interacting System At z$\sim$2. 2015, \textit{Astron. J.}, 149, 107
        (\href{http://arxiv.org/abs/1410.0967}{arXiv:1410.0967})
    \item[6]\hypertarget{15.wang.apj}{} \textbf{Wang, X.} et al. The Grism Lens-Amplified Survey from Space (GLASS) IV. 
        Mass reconstruction of the lensing cluster Abell 2744 from frontier field imaging and GLASS spectroscopy. 2015, 
        \textit{Astrophys. J.}, 811, 29 (\href{http://arxiv.org/abs/1504.02405}{arXiv:1504.02405})
    \item[7]\hypertarget{17.wang.apj}{} \textbf{Wang, X.} et al. The Grism Lens-Amplified Survey from Space (GLASS) X.
        Sub-kiloparsec resolution gas-phase metallicity maps at cosmic noon behind the Hubble Frontier Fields cluster MACS1149.6+2223. 
        2017, \textit{Astrophys. J.}, 837, 89 (\href{http://arxiv.org/abs/1610.07558}{arXiv:1610.07558})
\end{list}  \\
%\multicolumn{1}{c}{\textsc{Conference Contributions, Unrefereed}}    \\
%\vspace{-0.4cm}
%\begin{list}{\labelitemi}{\leftmargin=0.5em}
%    \item[5]\hypertarget{5}{} \textbf{Wang, X.}, \& Huang, Y. F. On the Counter-jet Emission in GRB Afterglows. 2010b, \textit{AIP Conference Proceedings}, 1279, 460 (\href{http://arxiv.org/abs/1012.0521}{arXiv:1012.0521})
%\end{list}   \\

\multicolumn{1}{c}{\textsc{Contributing Author Papers in Refereed Academic Journals}}      \\
\vspace{-0.2cm}
\begin{list}{\labelitemi}{\leftmargin=0.5em}
    \item[1]\hypertarget{12.meng.apj}{} Meng, X.-L., Zhang, T.-J., Zhan, H., \& \textbf{Wang, X.} Morphology of Galaxy Clusters: A Cosmological Model-Independent Test of the Cosmic Distance-Duality Relation. 2012, \textit{Astrophys. J.}, 745, 98 (\href{http://arxiv.org/abs/1104.2833}{arXiv:1104.2833})
    \item[2]\hypertarget{14.schmidt.apjl}{} Schmidt, K. B., Treu, T., Brammer, G. B., Bradac, M., \textbf{Wang, X.} et al. Through the Looking GLASS: HST Spectroscopy of Faint Galaxies Lensed by the Frontier Fields Cluster MACSJ0717.5+3745. 2014, \textit{Astrophys. J. Letters}, 782L, 36 (\href{http://arxiv.org/abs/1401.0532}{arXiv:1401.0532})
    \item[3]\hypertarget{}{} Treu, T., Schmidt, K. B., Brammer, G. B., Vulcani, B., \textbf{Wang, X.} et al. The Grism 
    Lens-Amplified Survey from Space (GLASS). I. Survey Overview and First Data Release, 2015, \textit{Astrophys. J.}, 812, 114 
    (\href{https://arxiv.org/abs/1509.00475}{arXiv:1509.00475})
    \item[4]\hypertarget{}{} Morishita, T., Abramson, L. E., Treu, T., Schmidt, K. B., Vulcani, B., \textbf{Wang, X.} 
    Characterizing Intracluster Light in the Hubble Frontier Fields. 2017, \textit{Astrophys. J.}, 846, 139 
    (\href{https://arxiv.org/abs/1610.08503}{arXiv:1610.08503})
    \item[5]\hypertarget{}{} Kelly, P. L., ..., \textbf{Wang, X.} et al. An individual star at redshift 1.5 extremely magnified by 
    a galaxy-cluster lens. 2018, \textit{Nature Astronomy}, in press (\href{https://arxiv.org/abs/1706.10279}{arXiv:1706.10279})
    \item[6]\hypertarget{}{} Morishita, T., Abramson, L. E., Treu, T., \textbf{Wang, X.} et al.  Metal Deficiency in Two Massive Dead 
    Galaxies at z$\sim$2, 2018, \textit{Astrophys. J. Letters}, 856L, 4 (\href{https://arxiv.org/abs/1803.01852}{arXiv:1803.01852})
\end{list}

\end{longtable}


%%%%%%%%%%%%%%%%%%%%%%%%%%%%%%%%%%%%%%%%%%%%%%%%%%%%%%%%%%%%%%%%%%%%%%%%%%%%%%
\vspace{-1.5em}
\section{Academic Activities (selected)}

\begin{longtable}{r|p{5.5in}}
%\begin{longtable}{r|p{16.5cm}}

    \textsc{Apr. 2009}   &   \textbf{Contributed talk}, @ \href{http://www.ns-grb.com/index0.html}{Frontiers of Space Astrophysics: Neutron Stars \& Gamma Ray Bursts --- Recent Developments \& Future Directions}, Cairo \& Alexandria, Egypt     \\
    \multicolumn{2}{c}{} \\

    \textsc{Jun. 2010}   &   \textbf{Contributed talk}, @ \href{http://www.physics.hku.hk/~astro/2010Astro/Index.htm}{A mini-workshop on ``Gamma-ray Sky from Fermi: Neutron Stars and their Environment''}, University of Hong Kong, Hong Kong   \\
    \multicolumn{2}{c}{} \\

%    \textsc{Nov. 2012}   &   \textbf{Presented a talk}, @ \href{http://www.phys.tsinghua.edu.cn/publish/phy/5287/2012/20121102084855753317440/20121102084855753317440_.html}{Tsinghua Transient Workshop 2012}, Tsinghua University, Beijing   \\
%            &   \emph{Title: Observational constraints on cosmic neutrinos and dark energy revisited}    \\
%    \multicolumn{2}{c}{} \\

    \textsc{Aug. 2015}   &   \textbf{Contributed talk}, @ \href{http://hffiau.epfl.ch/page-116896.html}{Focus Meeting 22 at XXIX IAU
General Assembly}, Honolulu, HI     \\
    \multicolumn{2}{c}{} \\

    \textsc{Jun. 2016}   &   \textbf{Invited talk}, @ Nanjing University, Nanjing \\
    \multicolumn{2}{c}{} \\

    \textsc{Jun. 2016}   &   \textbf{Invited talk}, @ Purple Mountain Observatory, Nanjing \\
    \multicolumn{2}{c}{} \\

%    \textsc{Jun. 2016}   &   \textbf{Invited talk}, @ Tsinghua University, Beijing \\
%    \multicolumn{2}{c}{} \\

    \textsc{Jun. 2016}   &   \textbf{Invited talk}, @ National Astronomical Observatories of China, Beijing   \\
    \multicolumn{2}{c}{} \\

    \textsc{Aug. 2016}   &   \textbf{Colloquium talk}, @ Department of Astronomy, University of Michigan, Ann Arbor, MI \\
    \multicolumn{2}{c}{} \\

    \textsc{Jan. 2017}   &   \textbf{Colloquium talk}, @ Steward Observatory, University of Arizona, Tucson, AZ \\
    \multicolumn{2}{c}{} \\

    \textsc{Jun. 2017}   &   \textbf{Contributed talk}, @ 
    \href{http://eas.unige.ch/EWASS2017/session.jsp?id=SS11}{Special Session 11 at 
    European Week of Astronomy and Space Science}, Prague, Czech Republic   \\
    \multicolumn{2}{c}{} \\

    \textsc{Aug. 2017}   &   \textbf{Contributed talk}, @
    \href{http://darkuniverse2017.csp.escience.cn/dct/page/65580}{Shedding Light on 
    the Dark Universe with Extremely Large Telescopes}, Lanzhou, China      \\
    \multicolumn{2}{c}{} \\

    \textsc{Sept. 2017}   &   \textbf{Invited talk}, @ Shanghai Jiao Tong University, Shanghai \\
    \multicolumn{2}{c}{} \\

    \textsc{Sept. 2017}   &   \textbf{Invited talk}, @ Nanjing University, Nanjing \\
    \multicolumn{2}{c}{} \\

    \textsc{Sept. 2017}   &   \textbf{Invited talk}, @ Tsinghua University, Beijing \\
    \multicolumn{2}{c}{} \\

    \textsc{Jan. 2018}   &   \href{http://obs.carnegiescience.edu/talk_event/828}{\textbf{Colloquium talk}}, @ Carnegie Observatories, Pasadena, CA \\
    \multicolumn{2}{c}{} \\

    \textsc{Feb. 2018}   &   \href{https://www.ipac.caltech.edu/event/358}{\textbf{Colloquium talk}}, @ IPAC, Caltech, Pasadena, CA \\
    \multicolumn{2}{c}{}

\end{longtable}


%%%%%%%%%%%%%%%%%%%%%%%%%%%%%%%%%%%%%%%%%%%%%%%%%%%%%%%%%%%%%%%%%%%%%%%%%%%%%%
\vspace{-1.2em}
\section{Awards and Honors (selected)}
\begin{tabular}{rp{5in}}
~~~~~~\textsc{Apr. 2015} & AAS International Travel Grant (\$1k)    \\
~~~~~~\textsc{Jun. 2014} & 1$^{\rm st}$ Prize for Excellent M.Sc. Thesis amongst all Universities and Colleges in Jiangsu Province   \\
~~~~~~\textsc{Sept. 2013} & Broida Fellowship, UCSB (\$3k)  \\
~~~~~~\textsc{Dec. 2012} & National Scholarship for Graduates ($\sim$\$4k)  \\
& {\it\small ~~~highest honorific scholarship in China conferred annually on excellent graduate students}\\
%~~~~~~\textsc{Jan. 2012} & Excellent Graduate Leader in School of Astronomy and Space Sciences, Nanjing University  \\
%~~~~~~\textsc{Nov. 2010} & Scholarship for Excellent Graduate Students in Nanjing University \\
~~~~~~\textsc{Aug. 2010} & 1$^{\rm st}$ Prize for Excellent B.Sc. Thesis amongst all Universities and Colleges in Jiangsu Province   \\
~~~~~~\textsc{Oct. 2009} & Scholarship of National Astronomical Observatories, Chinese Academy of Sciences   \\
%\textsc{2007--2008} & Special Scholarship for Innovation, People's Scholarship, \textit{Der} Scholarship, etc.   \\
\end{tabular}


%%%%%%%%%%%%%%%%%%%%%%%%%%%%%%%%%%%%%%%%%%%%%%%%%%%%%%%%%%%%%%%%%%%%%%%%%%%%%%
\vspace{0.8em}
\section{Computer Skills}
~~~~Python, MATLAB, FORTRAN, C, \LaTeX, vim, Github, Mathmatica


%%%%%%%%%%%%%%%%%%%%%%%%%%%%%%%%%%%%%%%%%%%%%%%%%%%%%%%%%%%%%%%%%%%%%%%%%%%%%%
\vspace{0.8em}
\section{Working Experience and Outreach Activities}
\begin{tabular}{rp{5.5in}}
%\begin{tabular}{rp{16cm}}

\textsc{2010--2012}  & President of Graduate Student Union in School of Astronomy and Space Sciences, NJU \\
\multicolumn{2}{c}{} \\

\textsc{\small Sept.--Dec. 2010}  & Teaching assistant of Theoretical Astrophysics (upper division undergraduate course), NJU \\
\multicolumn{2}{c}{} \\

\textsc{Dec. 2010}--  & Organizer of Graduate Journal Club in School of Astronomy 
and Space Sciences, NJU \\
\textsc{Dec. 2011}    & \small{In total, I arranged 17 meetings, and invited 
34 speakers.
%most of which are graduate students. The majority of the speakers come from our school, while we do have speakers from many other institutes, e.g., Purple Mountain Observatory, University of Science and Technology of China, University of Sydney. 
The topics are related to the major field of interest of the speakers, who will also share with participants some academic experience in doing scientific research. This activity is financially supported by our school.}  \\
\multicolumn{2}{c}{} \\

\textsc{\small Sept.--Dec. 2013}  & Teaching assistant of Physics Lab hands-on courses, UCSB   \\
\multicolumn{2}{c}{} \\
\textsc{2014--2015}  & Organizer of Treu Group Meetings, UCSB \& UCLA \\
\multicolumn{2}{c}{} \\
\textsc{2015--2017}  & Volunteer in the annual \textsc{Exploring Your Universe!} events, UCLA \\
\multicolumn{2}{c}{} \\
\textsc{2015--2017}  & Demonstrator of Astronomy experiments to local K12 schools in Los Angeles \\
\multicolumn{2}{c}{} \\


\end{tabular}


\end{document}
%--------------------------------------------------------------------------------------------------------------------------------
%                                                               END
%--------------------------------------------------------------------------------------------------------------------------------
%#############################################################################################################
% below are dropouts from previous versions of my cv
%#############################################################################################################

%--------------------------------------------------------
\textsc{Sept. 2012}--  &   \emph{Title: Applications of the Non-Parametric Bayesian Method to Cosmological Reconstructions}	\\
\textsc{Current}      &   \small{We attempt to use the newly proposed non-parametric Bayesian method to reconstruct the
evolution history of some crucial cosmological quantities, i.e., primordial perturbation power spectrum, dark energy equation of
state, growth function. We believe that some unique abrupt features of the primordial power spectrum can be revealed by our
method. With the help of Fisher formalism and principle component analysis, it is straightforward to forecast the possibility
that those features can be detected by next generation projects, i.e., AS3, BigBOSS, Euclid, CMBPol.}	\\
&   \textbf{Project in progress} \\
\multicolumn{2}{c}{} \\
%--------------------------------------------------------

\textsc{Jan. 2010}--     &   \emph{Title: Studies on the Temporal Analysis of Gamma-Ray Bursts}\\
\textsc{Jul. 2010}       &   \small{I review the observational characteristics of GRBs and demonstrate in detail the procedures of Swift/BAT data reduction. Using the method and shell scripts developed by myself, I have analyzed the data of all Swift GRBs with redshift measurements and derived spectral lags with high precision. I have tested some empirical relations between spectral lag and some observational quantities, i.e., redshift, T90, peak flux, hardness ratio, $\rm E_{peak}$, $\rm E_{iso}$, etc.. Moreover, I have proved there is a high possibility that spectral lags and isotropic luminosity of GRBs are correlated.} \\
                &   \textbf{Scientific Product: As my dissertation for B.Sc., this thesis has won 1$^{\rm st}$ prize for excellent B.Sc. thesis amongst all universities and colleges in Jiangsu province.} \\
\multicolumn{2}{c}{} \\
